\documentclass[12pt]{article}
\usepackage[margin=1in]{geometry} 
\input{comunes/preamble}
\usepackage{lastpage}
\usepackage{fancyhdr}
\pagestyle{fancy}
\setlength{\headheight}{42pt}

 
\begin{document}
\lhead{Ingeniería Física \\ Escuela de Física \\ Tecnológico de Costa Rica} 
\rhead{Instrumentación I \\ \, \\ Taller de simulación} 
\cfoot{\thepage\ de \pageref{LastPage}}

Realice lo siguiente:
\begin{enumerate}
\item Para el circuito de la Figura \ref{fig:T1F1} obtenga la potencia de la fuente de \SI{3}{\volt}, la corriente en la fuente de \SI{4}{\volt} y el voltaje en la resistencia de \SI{6}{\ohm}; para cada uno de los siguientes valores de la resistencia $R$:

\begin{tabularx}{\linewidth}{XXXX}
    \textbullet{Un corto} & \textbullet{Un abierto} & \textbullet{\SI{1}{\ohm}} & \textbullet{\SI{10}{\ohm}}
\end{tabularx}

Directivas Spice a utilizar: \verb|.tran|; \verb|.step| y \verb|.meas|.


\begin{figure}[H]
    \centering
    \begin{circuitikz}[scale=1.2] \draw
        (0,0) 	to[american voltage source, l=3\si{\volt}] (0,2)
        		to[resistor, l=2\si{\ohm}] (2,2) -- (6,2)
        		to[resistor, l_=5\si{\ohm}] (6,0) 
        		to[resistor, l_=2\si{\ohm}] (4,0)
        		to[resistor, l=1\si{\ohm}] (4,2)
        (0,0)   to[resistor, l=4\si{\ohm},i<_=$I_x$] (2,0)
        		to[american current source, l={0,2}\si{\ampere}] (2,2)
        (4,0)	to[american voltage source, l_=5\si{\volt}] (2,0)
        (6,2)	to[american controlled current source, l=$I_x$] (8,2)
        		to[resistor, l_=1\si{\ohm}] (8,0)
        		to[resistor, l_=3\si{\ohm}] (6,0)
        (8,2) 	to[resistor, l=3\si{\ohm}] (10,2) -- (10,-2)
        		to[resistor, l_=$R$] (6,-2)
        		to[resistor, l_=4\si{\ohm}] (4,-2)
        (2,-2)	to[american voltage source, l=4\si{\volt}] (4,-2)
        (6,-2) -- (6,0)
        (0,0)	to[resistor, l=6\si{\ohm}] (0,-4)
        (2,0)	-- (2,-4) -- (0,-4)
        (2,-4) 	to[resistor, l=2\si{\ohm}] (8,-4)
        (8,0)
        		to[resistor, l=5\si{\ohm}] (10,0)
        (10,-2) to[short] (10,-4) -- (8,-4)
        ;
    \end{circuitikz}
    \caption{Primer ejercicio de simulación}
    \label{fig:T1F1}
\end{figure}

\item Para el circuito de la Figura \ref{fig:T1F2} encuentre $V_a$. Directivas Spice a utilizar: \verb|.op| y \verb|.meas|.

\begin{figure}[H]
    \centering
    \begin{circuitikz}[scale=1] \draw
        (0,0)
            to[V,l=\SI{5}{\volt}]
        (0,2)
            to[cV,l=$5 V_a$]
        (0,4)
            to[R,l=\SI{5}{\ohm}]
        (3,4)
            to[R,l=\SI{5}{\ohm},v=$V_a$]
        (6,4)
            to[R,l=\SI{5}{\ohm}]
        (6,2)
            to[V, invert, l=\SI{10}{\volt}]
        (6,0)
            to[short]
        (0,0)
        (3,0)
            to[I,l=\SI{1}{\ampere}]
        (3,4)
        ;
    \end{circuitikz}
    \caption{Segundo ejercicio de simulación}
    \label{fig:T1F2}
\end{figure}

\item Para el circuito de la Figura \ref{fig:T1F3} encuentre el valor de $R$ que consuma la máxima potencia. Directivas Spice a utilizar: \verb|.op|; \verb|.step| y \verb|.meas|.

\begin{figure}[H]
    \centering
    \begin{circuitikz}[scale=1] \draw
        (0,0)
            to[V,l=\SI{20}{\volt}]
        (0,4)
            to[R,l=\SI{30}{\ohm}]
        (3,4)
            to[R,l=$R$]
        (6,4)
            to[R,l=\SI{40}{\ohm}]
        (6,2)
            to[V, invert, l=\SI{10}{\volt}]
        (6,0)
            to[short]
        (0,0)
        (3,0)
            to[I,l=\SI{0.5}{\ampere}]
        (3,4)
        ;
    \end{circuitikz}
    \caption{Tercer ejercicio de simulación}
    \label{fig:T1F3}
\end{figure}

\item Para el circuito de la Figura \ref{fig:T1F4} encuentre el valor de la fuente $V_e$ de tal forma que $V_x$ sea \SI{15}{\volt} y encuentre la resistencia equivalente del circuito visto desde $V_e$. Directivas Spice a utilizar: \verb|.op|; \verb|.step| y \verb|.meas|.

\begin{figure}[H]
    \begin{circuitikz}[scale=1.2] \draw
        (0,0)
            to[V,l=$V_e$]
        (0,2)
            to[R,l=\SI{30}{\ohm}]
        (2,2)
            to[R,l=\SI{20}{\ohm}]
        (4,2)
            to[R,l=\SI{50}{\ohm}]
        (6,2)
            to[R,l=\SI{25}{\ohm}]
        (6,0)
            to[R,l_=\SI{35}{\ohm}]
        (4,0)
            to[R,l=\SI{30}{\ohm}]
        (2,0)
            to[short]
        (0,0)
        (2,2)
            to[R,l_=\SI{10}{\ohm},v^=$V_x$]
        (2,0)
        (4,0)
            to[R,l=\SI{70}{\ohm}]
        (4,2)
        (4,0)
            to[R,l=\SI{30}{\ohm}]
        (6,2)
        (2,2)
            to[short]
        (2,4)
            to[short]
        (4,4)
            to[R,l=\SI{16}{\ohm}]
        (8,4)
            to[short]
        (8,-2)
            to[R,l_=\SI{40}{\ohm}]
        (4,-2)
            to[R,l_=\SI{15}{\ohm}]
        (2,-2)
            to[short]
        (2,0)
        (4,2)
            to[R,l=\SI{30}{\ohm}]
        (4,4)
        (4,-2)
            to[short]
        (4,0)
        (4,-1) 
            to[R,l=\SI{60}{\ohm}]
        (8,-1)
        ;
    \end{circuitikz}
\caption{Cuarto ejercicio de simulación}
\label{fig:T1F4}
\end{figure}

\end{enumerate}
\end{document}