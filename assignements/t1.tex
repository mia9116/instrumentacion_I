\documentclass[12pt]{article}
\usepackage[margin=1in]{geometry} 
\input{comunes/preamble}
\usepackage{lastpage}
\usepackage{fancyhdr}
\usepackage{csvsimple,booktabs}
\pagestyle{fancy}
\setlength{\headheight}{42pt}
\usepackage{caption}
 
\begin{document}
\lhead{Ingeniería Física \\ Escuela de Física \\ Tecnológico de Costa Rica} 
\rhead{Instrumentación II \\ Tarea \#1  \\ Entrega: Semana 3} 
\cfoot{\thepage\ de \pageref{LastPage}}
\setlength{\parindent}{0em}

Usando un Notebook de \href{https://colab.research.google.com/}{Google Colab} realice lo siguiente:

\begin{itemize}
    \item Lea el archivo \href{https://estudianteccr-my.sharepoint.com/:u:/g/personal/prof_juan_rojas_estudiantec_cr/EcuXJs2cG21HnH02L5fq5OMBOoznw5P7fMkWscsfJdJjgQ?e=8s1NAe}{adc.csv} que contiene los datos mostrados en la siguiente gráfica.
    \begin{figure}[H]
        \centering
        \begin{tikzpicture}
            \begin{axis}
            \addplot table [x=t, y=s, col sep=comma, mark=none] {data/adc.csv};
            \end{axis}
        \end{tikzpicture}
        \caption{}
        \label{fig:signal}
    \end{figure}
    \item Tomando en cuenta la teoría vista en clase, realice la cuantización y codificación de esta señal usando una implementación propia de \emph{Successive Approximation Register} (SAR) con resolución de 2bits y otra con 4 bits. El rango es de 0 a 5\si{V}
    \item Grafique la señal original y sus dos cuantizaciones en la misma gráfica.
    \item Calcule el error de cuantización $e_q$ en ambos casos
    \item Suba un enlace a su archivo de Colab
\end{itemize}

Cualquier entrega tardía se califica en base a 70. 

% \bibliographystyle{IEEEtran}
% \bibliography{ref_tareas}

\end{document}