\documentclass[12pt]{article}
\usepackage[margin=1in]{geometry} 
\input{comunes/preamble}
\usepackage{lastpage}
\usepackage{fancyhdr}
\usepackage{csvsimple,booktabs}
\pagestyle{fancy}
\setlength{\headheight}{42pt}
\usepackage{caption}
 
\begin{document}
\lhead{Ingeniería Física \\ Escuela de Física \\ Tecnológico de Costa Rica} 
\rhead{Instrumentación II \\ Tarea \#1  \\ Entrega: Semana 3} 
\cfoot{\thepage\ de \pageref{LastPage}}
\setlength{\parindent}{0em}

Usando un Notebook de \href{https://colab.research.google.com/}{Google Colab} realice lo siguiente:

\begin{itemize}
    \item Replique el algoritmo mostrado en la Figura \ref{fig:while} usando Python. Use como valor de \emph{\# de grupo} el numero correspondiente a su grupo.
    \begin{figure}[H]
        \centering
        \includegraphics[width=10cm]{fig/t1.1.png}
        \caption{}
        \label{fig:while}
    \end{figure}
    \item Imprima el valor de $x$, $y$ e $iter$ en la terminal una vez terminada la ejecución del bucle While
    \item Descarge el Notebook (.ipynb) y subalo al TecDigital
\end{itemize}

Cualquier entrega tardía se califica en base a 70. 


% \bibliographystyle{IEEEtran}
% \bibliography{ref_tareas}

\end{document}