\documentclass[12pt]{article}
\usepackage[margin=1in]{geometry} 
\input{comunes/preamble}
\usepackage{lastpage}
\usepackage{fancyhdr}
\usepackage{csvsimple,booktabs}
\pagestyle{fancy}
\setlength{\headheight}{42pt}
\usepackage{caption}
 
\begin{document}
\lhead{Ingeniería Física \\ Escuela de Física \\ Tecnológico de Costa Rica} 
\rhead{Instrumentación II \\ Tarea \#2  \\ Entrega: Semana 6} 
\cfoot{\thepage\ de \pageref{LastPage}}
\setlength{\parindent}{0em}

El diagrama de bloques mostrado en la Figura \ref{fig:for} es utilizado para generar -partir de enteros del tipo \verb+uint8+- un arreglo de cadenas de texto que incluye los números del 0 al 9 y las letras de la $A$ a la $Z$


\begin{figure}[H]
    \centering
    \includegraphics[width=15cm]{fig/t2.1.png}
    \caption{}
    \label{fig:for}
\end{figure}


Usando Simulink realice lo siguiente:

\begin{itemize}
    \item Descargue los archivos \href{https://github.com/juanjorojash/instrumentacion_II/blob/5b8d70f17f7545a45cf6b1aebac0eb5c55f2a2ad/simulaciones/t2.slx}{t2.slx} y \href{https://github.com/juanjorojash/instrumentacion_II/blob/3795263c9e017dead230c0d5c829228d59253f76/simulaciones/t2_process.m}{t2\_process.m} y estúdielos para entender su función.
    \item Modifique el diagrama de bloques de forma que genere un arreglo de cadenas de texto que incluya los números del 0 al 9, las letras de la $A$ a la $Z$ y las letras de la $a$ a la $z$, recuerde modificar el tiempo de la simulación.
    \item Usando \verb+t2_process.m+ genere el archivo \verb+datos.csv+ que contiene el resultado del arreglo separado por comas.
    \item Guarde el diagrama de bloques modificado con el nombre \verb+t2_sol.slx+
    \item Incluya los dos archivos en un archivo comprimido \verb+t2.zip+ y suba al TecDigital
\end{itemize}

Cualquier entrega tardía se califica en base a 70. 


% \bibliographystyle{IEEEtran}
% \bibliography{ref_tareas}

\end{document}